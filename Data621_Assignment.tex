\documentclass[]{article}
\usepackage{lmodern}
\usepackage{amssymb,amsmath}
\usepackage{ifxetex,ifluatex}
\usepackage{fixltx2e} % provides \textsubscript
\ifnum 0\ifxetex 1\fi\ifluatex 1\fi=0 % if pdftex
  \usepackage[T1]{fontenc}
  \usepackage[utf8]{inputenc}
\else % if luatex or xelatex
  \ifxetex
    \usepackage{mathspec}
  \else
    \usepackage{fontspec}
  \fi
  \defaultfontfeatures{Ligatures=TeX,Scale=MatchLowercase}
\fi
% use upquote if available, for straight quotes in verbatim environments
\IfFileExists{upquote.sty}{\usepackage{upquote}}{}
% use microtype if available
\IfFileExists{microtype.sty}{%
\usepackage{microtype}
\UseMicrotypeSet[protrusion]{basicmath} % disable protrusion for tt fonts
}{}
\usepackage[margin=1in]{geometry}
\usepackage{hyperref}
\hypersetup{unicode=true,
            pdftitle={Data621\_Assignment2},
            pdfauthor={Ritesh Lohiya},
            pdfborder={0 0 0},
            breaklinks=true}
\urlstyle{same}  % don't use monospace font for urls
\usepackage{color}
\usepackage{fancyvrb}
\newcommand{\VerbBar}{|}
\newcommand{\VERB}{\Verb[commandchars=\\\{\}]}
\DefineVerbatimEnvironment{Highlighting}{Verbatim}{commandchars=\\\{\}}
% Add ',fontsize=\small' for more characters per line
\usepackage{framed}
\definecolor{shadecolor}{RGB}{248,248,248}
\newenvironment{Shaded}{\begin{snugshade}}{\end{snugshade}}
\newcommand{\KeywordTok}[1]{\textcolor[rgb]{0.13,0.29,0.53}{\textbf{#1}}}
\newcommand{\DataTypeTok}[1]{\textcolor[rgb]{0.13,0.29,0.53}{#1}}
\newcommand{\DecValTok}[1]{\textcolor[rgb]{0.00,0.00,0.81}{#1}}
\newcommand{\BaseNTok}[1]{\textcolor[rgb]{0.00,0.00,0.81}{#1}}
\newcommand{\FloatTok}[1]{\textcolor[rgb]{0.00,0.00,0.81}{#1}}
\newcommand{\ConstantTok}[1]{\textcolor[rgb]{0.00,0.00,0.00}{#1}}
\newcommand{\CharTok}[1]{\textcolor[rgb]{0.31,0.60,0.02}{#1}}
\newcommand{\SpecialCharTok}[1]{\textcolor[rgb]{0.00,0.00,0.00}{#1}}
\newcommand{\StringTok}[1]{\textcolor[rgb]{0.31,0.60,0.02}{#1}}
\newcommand{\VerbatimStringTok}[1]{\textcolor[rgb]{0.31,0.60,0.02}{#1}}
\newcommand{\SpecialStringTok}[1]{\textcolor[rgb]{0.31,0.60,0.02}{#1}}
\newcommand{\ImportTok}[1]{#1}
\newcommand{\CommentTok}[1]{\textcolor[rgb]{0.56,0.35,0.01}{\textit{#1}}}
\newcommand{\DocumentationTok}[1]{\textcolor[rgb]{0.56,0.35,0.01}{\textbf{\textit{#1}}}}
\newcommand{\AnnotationTok}[1]{\textcolor[rgb]{0.56,0.35,0.01}{\textbf{\textit{#1}}}}
\newcommand{\CommentVarTok}[1]{\textcolor[rgb]{0.56,0.35,0.01}{\textbf{\textit{#1}}}}
\newcommand{\OtherTok}[1]{\textcolor[rgb]{0.56,0.35,0.01}{#1}}
\newcommand{\FunctionTok}[1]{\textcolor[rgb]{0.00,0.00,0.00}{#1}}
\newcommand{\VariableTok}[1]{\textcolor[rgb]{0.00,0.00,0.00}{#1}}
\newcommand{\ControlFlowTok}[1]{\textcolor[rgb]{0.13,0.29,0.53}{\textbf{#1}}}
\newcommand{\OperatorTok}[1]{\textcolor[rgb]{0.81,0.36,0.00}{\textbf{#1}}}
\newcommand{\BuiltInTok}[1]{#1}
\newcommand{\ExtensionTok}[1]{#1}
\newcommand{\PreprocessorTok}[1]{\textcolor[rgb]{0.56,0.35,0.01}{\textit{#1}}}
\newcommand{\AttributeTok}[1]{\textcolor[rgb]{0.77,0.63,0.00}{#1}}
\newcommand{\RegionMarkerTok}[1]{#1}
\newcommand{\InformationTok}[1]{\textcolor[rgb]{0.56,0.35,0.01}{\textbf{\textit{#1}}}}
\newcommand{\WarningTok}[1]{\textcolor[rgb]{0.56,0.35,0.01}{\textbf{\textit{#1}}}}
\newcommand{\AlertTok}[1]{\textcolor[rgb]{0.94,0.16,0.16}{#1}}
\newcommand{\ErrorTok}[1]{\textcolor[rgb]{0.64,0.00,0.00}{\textbf{#1}}}
\newcommand{\NormalTok}[1]{#1}
\usepackage{longtable,booktabs}
\usepackage{graphicx,grffile}
\makeatletter
\def\maxwidth{\ifdim\Gin@nat@width>\linewidth\linewidth\else\Gin@nat@width\fi}
\def\maxheight{\ifdim\Gin@nat@height>\textheight\textheight\else\Gin@nat@height\fi}
\makeatother
% Scale images if necessary, so that they will not overflow the page
% margins by default, and it is still possible to overwrite the defaults
% using explicit options in \includegraphics[width, height, ...]{}
\setkeys{Gin}{width=\maxwidth,height=\maxheight,keepaspectratio}
\IfFileExists{parskip.sty}{%
\usepackage{parskip}
}{% else
\setlength{\parindent}{0pt}
\setlength{\parskip}{6pt plus 2pt minus 1pt}
}
\setlength{\emergencystretch}{3em}  % prevent overfull lines
\providecommand{\tightlist}{%
  \setlength{\itemsep}{0pt}\setlength{\parskip}{0pt}}
\setcounter{secnumdepth}{0}
% Redefines (sub)paragraphs to behave more like sections
\ifx\paragraph\undefined\else
\let\oldparagraph\paragraph
\renewcommand{\paragraph}[1]{\oldparagraph{#1}\mbox{}}
\fi
\ifx\subparagraph\undefined\else
\let\oldsubparagraph\subparagraph
\renewcommand{\subparagraph}[1]{\oldsubparagraph{#1}\mbox{}}
\fi

%%% Use protect on footnotes to avoid problems with footnotes in titles
\let\rmarkdownfootnote\footnote%
\def\footnote{\protect\rmarkdownfootnote}

%%% Change title format to be more compact
\usepackage{titling}

% Create subtitle command for use in maketitle
\newcommand{\subtitle}[1]{
  \posttitle{
    \begin{center}\large#1\end{center}
    }
}

\setlength{\droptitle}{-2em}
  \title{Data621\_Assignment2}
  \pretitle{\vspace{\droptitle}\centering\huge}
  \posttitle{\par}
  \author{Ritesh Lohiya}
  \preauthor{\centering\large\emph}
  \postauthor{\par}
  \predate{\centering\large\emph}
  \postdate{\par}
  \date{June 23, 2018}


\begin{document}
\maketitle

Overview In this homework assignment, you will work through various
classification metrics. You will be asked to create functions in R to
carry out the various calculations. You will also investigate some
functions in packages that will let you obtain the equivalent results.
Finally, you will create graphical output that also can be used to
evaluate the output of classification models, such as binary logistic
regression. \#

\begin{Shaded}
\begin{Highlighting}[]
\KeywordTok{library}\NormalTok{(caret)}
\end{Highlighting}
\end{Shaded}

\begin{verbatim}
## Loading required package: lattice
\end{verbatim}

\begin{verbatim}
## Loading required package: ggplot2
\end{verbatim}

\begin{Shaded}
\begin{Highlighting}[]
\KeywordTok{library}\NormalTok{(pROC)}
\end{Highlighting}
\end{Shaded}

\begin{verbatim}
## Type 'citation("pROC")' for a citation.
\end{verbatim}

\begin{verbatim}
## 
## Attaching package: 'pROC'
\end{verbatim}

\begin{verbatim}
## The following objects are masked from 'package:stats':
## 
##     cov, smooth, var
\end{verbatim}

\begin{Shaded}
\begin{Highlighting}[]
\KeywordTok{library}\NormalTok{(ggplot2)}
\end{Highlighting}
\end{Shaded}

\begin{enumerate}
\def\labelenumi{\arabic{enumi}.}
\tightlist
\item
  Download the classification output data set (attached in Blackboard to
  the assignment).
\end{enumerate}

\begin{Shaded}
\begin{Highlighting}[]
\NormalTok{data <-}\StringTok{ }\KeywordTok{read.csv}\NormalTok{(}\StringTok{"https://raw.githubusercontent.com/Riteshlohiya/Data621-Assignment2/master/classification-output-data.csv"}\NormalTok{, }\DataTypeTok{stringsAsFactors =} \OtherTok{FALSE}\NormalTok{, }\DataTypeTok{sep =} \StringTok{","}\NormalTok{, }\DataTypeTok{header =} \OtherTok{TRUE}\NormalTok{)}

\KeywordTok{summary}\NormalTok{(data)}
\end{Highlighting}
\end{Shaded}

\begin{verbatim}
##     pregnant         glucose        diastolic        skinfold   
##  Min.   : 0.000   Min.   : 57.0   Min.   : 38.0   Min.   : 0.0  
##  1st Qu.: 1.000   1st Qu.: 99.0   1st Qu.: 64.0   1st Qu.: 0.0  
##  Median : 3.000   Median :112.0   Median : 70.0   Median :22.0  
##  Mean   : 3.862   Mean   :118.3   Mean   : 71.7   Mean   :19.8  
##  3rd Qu.: 6.000   3rd Qu.:136.0   3rd Qu.: 78.0   3rd Qu.:32.0  
##  Max.   :15.000   Max.   :197.0   Max.   :104.0   Max.   :54.0  
##     insulin            bmi           pedigree           age       
##  Min.   :  0.00   Min.   :19.40   Min.   :0.0850   Min.   :21.00  
##  1st Qu.:  0.00   1st Qu.:26.30   1st Qu.:0.2570   1st Qu.:24.00  
##  Median :  0.00   Median :31.60   Median :0.3910   Median :30.00  
##  Mean   : 63.77   Mean   :31.58   Mean   :0.4496   Mean   :33.31  
##  3rd Qu.:105.00   3rd Qu.:36.00   3rd Qu.:0.5800   3rd Qu.:41.00  
##  Max.   :543.00   Max.   :50.00   Max.   :2.2880   Max.   :67.00  
##      class         scored.class    scored.probability
##  Min.   :0.0000   Min.   :0.0000   Min.   :0.02323   
##  1st Qu.:0.0000   1st Qu.:0.0000   1st Qu.:0.11702   
##  Median :0.0000   Median :0.0000   Median :0.23999   
##  Mean   :0.3149   Mean   :0.1768   Mean   :0.30373   
##  3rd Qu.:1.0000   3rd Qu.:0.0000   3rd Qu.:0.43093   
##  Max.   :1.0000   Max.   :1.0000   Max.   :0.94633
\end{verbatim}

\begin{enumerate}
\def\labelenumi{\arabic{enumi}.}
\setcounter{enumi}{1}
\tightlist
\item
  The data set has three key columns we will use: ??? class: the actual
  class for the observation scored.class: the predicted class for the
  observation (based on a threshold of 0.5) ??? scored.probability: the
  predicted probability of success for the observation. Use the table()
  function to get the raw confusion matrix for this scored dataset. Make
  sure you understand the output. In particular, do the rows represent
  the actual or predicted class? The columns?
\end{enumerate}

\begin{Shaded}
\begin{Highlighting}[]
\NormalTok{r <-}\StringTok{ }\KeywordTok{table}\NormalTok{(data}\OperatorTok{$}\NormalTok{scored.class,data}\OperatorTok{$}\NormalTok{class)}
\NormalTok{knitr}\OperatorTok{::}\StringTok{ }\KeywordTok{kable}\NormalTok{(r)}
\end{Highlighting}
\end{Shaded}

\begin{longtable}[]{@{}lrr@{}}
\toprule
& 0 & 1\tabularnewline
\midrule
\endhead
0 & 119 & 30\tabularnewline
1 & 5 & 27\tabularnewline
\bottomrule
\end{longtable}

\begin{enumerate}
\def\labelenumi{\arabic{enumi}.}
\setcounter{enumi}{2}
\tightlist
\item
  Write a function that takes the data set as a dataframe, with actual
  and predicted classifications identified, and returns the accuracy of
  the predictions.
\end{enumerate}

\begin{Shaded}
\begin{Highlighting}[]
\NormalTok{Accuracy <-}\StringTok{ }\ControlFlowTok{function}\NormalTok{(data) \{}
\NormalTok{tb =}\StringTok{ }\KeywordTok{table}\NormalTok{(data}\OperatorTok{$}\NormalTok{class,data}\OperatorTok{$}\NormalTok{scored.class)}
\NormalTok{TN=tb[}\DecValTok{1}\NormalTok{,}\DecValTok{1}\NormalTok{]}
\NormalTok{TP=tb[}\DecValTok{2}\NormalTok{,}\DecValTok{2}\NormalTok{]}
\NormalTok{FN=tb[}\DecValTok{2}\NormalTok{,}\DecValTok{1}\NormalTok{]}
\NormalTok{FP=tb[}\DecValTok{1}\NormalTok{,}\DecValTok{2}\NormalTok{]}
  
\KeywordTok{return}\NormalTok{((TP}\OperatorTok{+}\NormalTok{TN)}\OperatorTok{/}\NormalTok{(TP}\OperatorTok{+}\NormalTok{FP}\OperatorTok{+}\NormalTok{TN}\OperatorTok{+}\NormalTok{FN))}
  
\NormalTok{\}}
\KeywordTok{Accuracy}\NormalTok{(data)}
\end{Highlighting}
\end{Shaded}

\begin{verbatim}
## [1] 0.8066298
\end{verbatim}

\begin{enumerate}
\def\labelenumi{\arabic{enumi}.}
\setcounter{enumi}{3}
\tightlist
\item
  Write a function that takes the data set as a dataframe, with actual
  and predicted classifications identified, and returns the
  classification error rate of the predictions.
\end{enumerate}

\begin{Shaded}
\begin{Highlighting}[]
\NormalTok{CER <-}\StringTok{ }\ControlFlowTok{function}\NormalTok{(data) \{}
\NormalTok{tb =}\StringTok{ }\KeywordTok{table}\NormalTok{(data}\OperatorTok{$}\NormalTok{class,data}\OperatorTok{$}\NormalTok{scored.class)}
\NormalTok{TN=tb[}\DecValTok{1}\NormalTok{,}\DecValTok{1}\NormalTok{]}
\NormalTok{TP=tb[}\DecValTok{2}\NormalTok{,}\DecValTok{2}\NormalTok{]}
\NormalTok{FN=tb[}\DecValTok{2}\NormalTok{,}\DecValTok{1}\NormalTok{]}
\NormalTok{FP=tb[}\DecValTok{1}\NormalTok{,}\DecValTok{2}\NormalTok{]}
  
\KeywordTok{return}\NormalTok{((FP}\OperatorTok{+}\NormalTok{FN)}\OperatorTok{/}\NormalTok{(TP}\OperatorTok{+}\NormalTok{FP}\OperatorTok{+}\NormalTok{TN}\OperatorTok{+}\NormalTok{FN))}
  
\NormalTok{\}}

\KeywordTok{CER}\NormalTok{(data)}
\end{Highlighting}
\end{Shaded}

\begin{verbatim}
## [1] 0.1933702
\end{verbatim}

\begin{enumerate}
\def\labelenumi{\arabic{enumi}.}
\setcounter{enumi}{4}
\tightlist
\item
  Write a function that takes the data set as a dataframe, with actual
  and predicted classifications identified, and returns the precision of
  the predictions.
\end{enumerate}

\begin{Shaded}
\begin{Highlighting}[]
\NormalTok{Precision <-}\StringTok{ }\ControlFlowTok{function}\NormalTok{(data) \{}
\NormalTok{tb =}\StringTok{ }\KeywordTok{table}\NormalTok{(data}\OperatorTok{$}\NormalTok{class,data}\OperatorTok{$}\NormalTok{scored.class)}
\NormalTok{TP=tb[}\DecValTok{2}\NormalTok{,}\DecValTok{2}\NormalTok{]}
\NormalTok{FP=tb[}\DecValTok{1}\NormalTok{,}\DecValTok{2}\NormalTok{]}
  
\KeywordTok{return}\NormalTok{((TP)}\OperatorTok{/}\NormalTok{(TP}\OperatorTok{+}\NormalTok{FP))}
  
\NormalTok{\}}
\KeywordTok{Precision}\NormalTok{(data)}
\end{Highlighting}
\end{Shaded}

\begin{verbatim}
## [1] 0.84375
\end{verbatim}

\begin{enumerate}
\def\labelenumi{\arabic{enumi}.}
\setcounter{enumi}{5}
\tightlist
\item
  Write a function that takes the data set as a dataframe, with actual
  and predicted classifications identified, and returns the sensitivity
  of the predictions. Sensitivity is also known as recall.
\end{enumerate}

\begin{Shaded}
\begin{Highlighting}[]
\NormalTok{Sensitivity <-}\StringTok{ }\ControlFlowTok{function}\NormalTok{(data) \{}
\NormalTok{tb =}\StringTok{ }\KeywordTok{table}\NormalTok{(data}\OperatorTok{$}\NormalTok{class,data}\OperatorTok{$}\NormalTok{scored.class)}
\NormalTok{TP=tb[}\DecValTok{2}\NormalTok{,}\DecValTok{2}\NormalTok{]}
\NormalTok{FN=tb[}\DecValTok{2}\NormalTok{,}\DecValTok{1}\NormalTok{]}
  
\KeywordTok{return}\NormalTok{((TP)}\OperatorTok{/}\NormalTok{(TP}\OperatorTok{+}\NormalTok{FN))}
  
\NormalTok{\}}

\KeywordTok{Sensitivity}\NormalTok{(data)}
\end{Highlighting}
\end{Shaded}

\begin{verbatim}
## [1] 0.4736842
\end{verbatim}

\begin{enumerate}
\def\labelenumi{\arabic{enumi}.}
\setcounter{enumi}{6}
\tightlist
\item
  Write a function that takes the data set as a dataframe, with actual
  and predicted classifications identified, and returns the specificity
  of the predictions.
\end{enumerate}

\begin{Shaded}
\begin{Highlighting}[]
\NormalTok{Specificity <-}\StringTok{ }\ControlFlowTok{function}\NormalTok{(data) \{}
\NormalTok{tb =}\StringTok{ }\KeywordTok{table}\NormalTok{(data}\OperatorTok{$}\NormalTok{class,data}\OperatorTok{$}\NormalTok{scored.class)}
\NormalTok{TN=tb[}\DecValTok{1}\NormalTok{,}\DecValTok{1}\NormalTok{]}
\NormalTok{TP=tb[}\DecValTok{2}\NormalTok{,}\DecValTok{2}\NormalTok{]}
\NormalTok{FN=tb[}\DecValTok{2}\NormalTok{,}\DecValTok{1}\NormalTok{]}
\NormalTok{FP=tb[}\DecValTok{1}\NormalTok{,}\DecValTok{2}\NormalTok{]}

\KeywordTok{return}\NormalTok{((TN)}\OperatorTok{/}\NormalTok{(TN}\OperatorTok{+}\NormalTok{FP))}
  
\NormalTok{\}}
\KeywordTok{Specificity}\NormalTok{(data)}
\end{Highlighting}
\end{Shaded}

\begin{verbatim}
## [1] 0.9596774
\end{verbatim}

\begin{enumerate}
\def\labelenumi{\arabic{enumi}.}
\setcounter{enumi}{7}
\tightlist
\item
  Write a function that takes the data set as a dataframe, with actual
  and predicted classifications identified, and returns the F1 score of
  the predictions.
\end{enumerate}

\begin{Shaded}
\begin{Highlighting}[]
\NormalTok{F1_score <-}\StringTok{ }\ControlFlowTok{function}\NormalTok{(data) \{}
\NormalTok{tb =}\StringTok{ }\KeywordTok{table}\NormalTok{(data}\OperatorTok{$}\NormalTok{class,data}\OperatorTok{$}\NormalTok{scored.class)}
\NormalTok{TN=tb[}\DecValTok{1}\NormalTok{,}\DecValTok{1}\NormalTok{]}
\NormalTok{TP=tb[}\DecValTok{2}\NormalTok{,}\DecValTok{2}\NormalTok{]}
\NormalTok{FN=tb[}\DecValTok{2}\NormalTok{,}\DecValTok{1}\NormalTok{]}
\NormalTok{FP=tb[}\DecValTok{1}\NormalTok{,}\DecValTok{2}\NormalTok{]}
  
  
\NormalTok{Precision =}\StringTok{ }\NormalTok{(TP)}\OperatorTok{/}\NormalTok{(TP}\OperatorTok{+}\NormalTok{FP)}
\NormalTok{Sensitivity =}\StringTok{ }\NormalTok{(TP)}\OperatorTok{/}\NormalTok{(TP}\OperatorTok{+}\NormalTok{FN)}
\NormalTok{Precision =(TP)}\OperatorTok{/}\NormalTok{(TP}\OperatorTok{+}\NormalTok{FP)}
  
\KeywordTok{return}\NormalTok{((}\DecValTok{2}\OperatorTok{*}\NormalTok{Precision}\OperatorTok{*}\NormalTok{Sensitivity)}\OperatorTok{/}\NormalTok{(Precision}\OperatorTok{+}\NormalTok{Sensitivity))}
  
\NormalTok{\}}
\KeywordTok{F1_score}\NormalTok{(data)}
\end{Highlighting}
\end{Shaded}

\begin{verbatim}
## [1] 0.6067416
\end{verbatim}

\begin{enumerate}
\def\labelenumi{\arabic{enumi}.}
\setcounter{enumi}{8}
\tightlist
\item
  Before we move on, let's consider a question that was asked: What are
  the bounds on the F1 score? Show that the F1 score will always be
  between 0 and 1. (Hint: If 0 \textless{} ???? \textless{} 1 and 0
  \textless{} ???? \textless{} 1 then ???????? \textless{} ????.)
\end{enumerate}

Both Precision and Sensitivity used to calculate F1 score are bounded
between 0 and 1 , so F1 score will be between 0 and 1.

\begin{enumerate}
\def\labelenumi{\arabic{enumi}.}
\setcounter{enumi}{9}
\tightlist
\item
  Write a function that generates an ROC curve from a data set with a
  true classification column (class in our example) and a probability
  column (scored.probability in our example). Your function should
  return a list that includes the plot of the ROC curve and a vector
  that contains the calculated area under the curve (AUC). Note that I
  recommend using a sequence of thresholds ranging from 0 to 1 at 0.01
  intervals.
\end{enumerate}

\begin{Shaded}
\begin{Highlighting}[]
\NormalTok{ROC =}\StringTok{ }\ControlFlowTok{function}\NormalTok{(labels, scores)\{}
\NormalTok{  labels =}\StringTok{ }\NormalTok{labels[}\KeywordTok{order}\NormalTok{(scores, }\DataTypeTok{decreasing=}\OtherTok{TRUE}\NormalTok{)]}
\NormalTok{  result =}\KeywordTok{data.frame}\NormalTok{(}\DataTypeTok{TPR=}\KeywordTok{cumsum}\NormalTok{(labels)}\OperatorTok{/}\KeywordTok{sum}\NormalTok{(labels), }\DataTypeTok{FPR=}\KeywordTok{cumsum}\NormalTok{(}\OperatorTok{!}\NormalTok{labels)}\OperatorTok{/}\KeywordTok{sum}\NormalTok{(}\OperatorTok{!}\NormalTok{labels), labels)}
  
\NormalTok{  FPR_df =}\StringTok{ }\KeywordTok{c}\NormalTok{(}\KeywordTok{diff}\NormalTok{(result}\OperatorTok{$}\NormalTok{FPR), }\DecValTok{0}\NormalTok{)}
\NormalTok{  TPR_df =}\StringTok{ }\KeywordTok{c}\NormalTok{(}\KeywordTok{diff}\NormalTok{(result}\OperatorTok{$}\NormalTok{TPR), }\DecValTok{0}\NormalTok{)}
\NormalTok{  AUC =}\StringTok{ }\KeywordTok{round}\NormalTok{(}\KeywordTok{sum}\NormalTok{(result}\OperatorTok{$}\NormalTok{TPR }\OperatorTok{*}\StringTok{ }\NormalTok{FPR_df) }\OperatorTok{+}\StringTok{ }\KeywordTok{sum}\NormalTok{(TPR_df }\OperatorTok{*}\StringTok{ }\NormalTok{FPR_df)}\OperatorTok{/}\DecValTok{2}\NormalTok{,}\DecValTok{4}\NormalTok{)}

  \KeywordTok{plot}\NormalTok{(result}\OperatorTok{$}\NormalTok{FPR,result}\OperatorTok{$}\NormalTok{TPR,}\DataTypeTok{type=}\StringTok{"l"}\NormalTok{,}\DataTypeTok{main =}\StringTok{"ROC Curve"}\NormalTok{,}\DataTypeTok{ylab=}\StringTok{"Sensitivity"}\NormalTok{,}\DataTypeTok{xlab=}\StringTok{"1-Specificity"}\NormalTok{)}
  \KeywordTok{abline}\NormalTok{(}\DataTypeTok{a=}\DecValTok{0}\NormalTok{,}\DataTypeTok{b=}\DecValTok{1}\NormalTok{)}
  \KeywordTok{legend}\NormalTok{(.}\DecValTok{6}\NormalTok{,.}\DecValTok{2}\NormalTok{,AUC,}\DataTypeTok{title =} \StringTok{"AUC"}\NormalTok{)}
  
\NormalTok{\}}

\KeywordTok{ROC}\NormalTok{(data}\OperatorTok{$}\NormalTok{class,data}\OperatorTok{$}\NormalTok{scored.probability)}
\end{Highlighting}
\end{Shaded}

\includegraphics{Data621_Assignment_files/figure-latex/unnamed-chunk-10-1.pdf}

\begin{enumerate}
\def\labelenumi{\arabic{enumi}.}
\setcounter{enumi}{10}
\tightlist
\item
  Use your created R functions and the provided classification output
  data set to produce all of the classification metrics discussed above.
\end{enumerate}

\begin{Shaded}
\begin{Highlighting}[]
\KeywordTok{Accuracy}\NormalTok{(data)}
\end{Highlighting}
\end{Shaded}

\begin{verbatim}
## [1] 0.8066298
\end{verbatim}

\begin{Shaded}
\begin{Highlighting}[]
\KeywordTok{CER}\NormalTok{(data)}
\end{Highlighting}
\end{Shaded}

\begin{verbatim}
## [1] 0.1933702
\end{verbatim}

\begin{Shaded}
\begin{Highlighting}[]
\KeywordTok{Precision}\NormalTok{(data)}
\end{Highlighting}
\end{Shaded}

\begin{verbatim}
## [1] 0.84375
\end{verbatim}

\begin{Shaded}
\begin{Highlighting}[]
\KeywordTok{Sensitivity}\NormalTok{(data)}
\end{Highlighting}
\end{Shaded}

\begin{verbatim}
## [1] 0.4736842
\end{verbatim}

\begin{Shaded}
\begin{Highlighting}[]
\KeywordTok{Specificity}\NormalTok{(data)}
\end{Highlighting}
\end{Shaded}

\begin{verbatim}
## [1] 0.9596774
\end{verbatim}

\begin{Shaded}
\begin{Highlighting}[]
\KeywordTok{F1_score}\NormalTok{(data)}
\end{Highlighting}
\end{Shaded}

\begin{verbatim}
## [1] 0.6067416
\end{verbatim}

\begin{enumerate}
\def\labelenumi{\arabic{enumi}.}
\setcounter{enumi}{11}
\tightlist
\item
  Investigate the caret package. In particular, consider the functions
  confusionMatrix, sensitivity, and specificity. Apply the functions to
  the data set. How do the results compare with your own functions?
\end{enumerate}

\begin{Shaded}
\begin{Highlighting}[]
\KeywordTok{confusionMatrix}\NormalTok{(}\KeywordTok{as.factor}\NormalTok{(data}\OperatorTok{$}\NormalTok{scored.class), }\KeywordTok{as.factor}\NormalTok{(data}\OperatorTok{$}\NormalTok{class), }\DataTypeTok{positive =} \StringTok{"1"}\NormalTok{)}
\end{Highlighting}
\end{Shaded}

\begin{verbatim}
## Confusion Matrix and Statistics
## 
##           Reference
## Prediction   0   1
##          0 119  30
##          1   5  27
##                                           
##                Accuracy : 0.8066          
##                  95% CI : (0.7415, 0.8615)
##     No Information Rate : 0.6851          
##     P-Value [Acc > NIR] : 0.0001712       
##                                           
##                   Kappa : 0.4916          
##  Mcnemar's Test P-Value : 4.976e-05       
##                                           
##             Sensitivity : 0.4737          
##             Specificity : 0.9597          
##          Pos Pred Value : 0.8438          
##          Neg Pred Value : 0.7987          
##              Prevalence : 0.3149          
##          Detection Rate : 0.1492          
##    Detection Prevalence : 0.1768          
##       Balanced Accuracy : 0.7167          
##                                           
##        'Positive' Class : 1               
## 
\end{verbatim}

The results are similar.

\begin{enumerate}
\def\labelenumi{\arabic{enumi}.}
\setcounter{enumi}{12}
\tightlist
\item
  Investigate the pROC package. Use it to generate an ROC curve for the
  data set. How do the results compare with your own functions?
\end{enumerate}

\begin{Shaded}
\begin{Highlighting}[]
\KeywordTok{par}\NormalTok{(}\DataTypeTok{mfrow=}\KeywordTok{c}\NormalTok{(}\DecValTok{1}\NormalTok{,}\DecValTok{2}\NormalTok{))}
\KeywordTok{plot}\NormalTok{(}\KeywordTok{roc}\NormalTok{(data}\OperatorTok{$}\NormalTok{class,data}\OperatorTok{$}\NormalTok{scored.probability),}\DataTypeTok{print.auc=}\OtherTok{TRUE}\NormalTok{)}
\KeywordTok{ROC}\NormalTok{(data}\OperatorTok{$}\NormalTok{class,data}\OperatorTok{$}\NormalTok{scored.probability)}
\end{Highlighting}
\end{Shaded}

\includegraphics{Data621_Assignment_files/figure-latex/unnamed-chunk-13-1.pdf}


\end{document}
